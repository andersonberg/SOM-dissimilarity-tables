%Apresentao ADS - CIn
%Anderson Berg - absd@cin.ufpe.br

\documentclass[pdf,t]{beamer}
\usetheme{CinArthur2}
%\usecolortheme{Beaver}

\usepackage{time}
\usepackage{graphicx}
\usepackage[T1]{fontenc}
\usepackage[brazil]{babel}
\usepackage{longtable}
\usepackage{amssymb,amsmath}
\usepackage{palatino}
\usepackage{xcolor}
\usepackage{hyperref}
\usepackage{enumerate}

\setbeamertemplate{itemize item}{$\bullet$}
\setbeamertemplate{itemize subitem}{$\checkmark$}
\setbeamertemplate{itemize subsubitem}{$\star$}

\providecommand{\abs}[1]{\lvert#1\rvert}
\providecommand{\norm}[1]{\lVert#1\rVert}

\begin{document}

% Informacoes para a Capa
\title[Adaptive batch SOM for multiple dissimilarity data tables]{Adaptive batch SOM for multiple dissimilarity data tables}%
%\subtitle {Uma Abordagem Simb�lica}
\author[Anderson Dantas]{ \Large{Anderson Dantas}\\
\footnotesize{absd@cin.ufpe.br}\\
%\author[Francisco de Carvalho]{ \Large{Francisco de Carvalho}\\
%\footnotesize{fatc@cin.ufpe.br}\\
\vspace*{0.25cm}
\includegraphics[height=2.0cm]{./figuras/ufpe}\\
\vspace*{0.5cm}
\includegraphics[height=1.4cm]{./figuras/logo-cin}
\date{\today}
}

% Capa
\frame{
  \titlepage
}

% Indice
%\AtBeginSection[]
%{
%  \begin{frame}
%    \frametitle{�ndice}
%    \scriptsize
%    \tableofcontents[currentsection,hideothersubsections]
%    \normalsize
% \end{frame}
%}

%\section{Batch Self-Organizing Map}

%\begin{frame}[c]
%\frametitle{Batch Self-Organizing Map Algorithm}
%Iterative two step algorithm: \\
%Affectation and representation steps
%\end{frame}

%\section{Batch Self-Organizing Map for dissimilarity data}

\begin{frame}[c]
\frametitle{Batch Self-Organizing Map Algorithm for dissimilarity data}
\begin{center}
Iterative two step algorithm: \\
Affectation and representation steps\\
\end{center}
\end{frame}

\begin{frame}[c]
\frametitle{Cost function}
\begin{center}
Extension of the k-means cost function
$$J = \sum_{i = 1}^n \sum_{r = 1}^m K^T(\delta(f(e_i),r)) d(e_i, g_r)$$
$$\lim_{|x| \to \infty} K(x) = 0$$
\end{center}
\end{frame}

\begin{frame}[c]
\frametitle{Generalised distance}
$$d^T(e_i, g_{f(e_i)}) = \sum_{r=1}^m K^T(\delta(f^T(e_i), r))d(e_i, g_r)$$
\end{frame}

\begin{frame}[c]
\frametitle{Affectation step}
\begin{center}
Function $f$ associates $e_i$ to the "closest" neuron
$$
c = f^T(e_i) = arg \min_{1 \leq r \leq m} d^T(e_i, g_r)
$$
\end{center}
\end{frame}

\begin{frame}[c]
\frametitle{Representation step}
\begin{center}
New prototypes are selected
$$
g^*_r = arg \min_{e \in E} \sum_{i=1}^n K^T (\delta(f^T(e_i),r)) d^T(e_i, e_r)
$$
\end{center}
\end{frame}

\begin{frame}[c]
\frametitle{Adaptive batch SOM for data based on multiple dissimilarity matrices}
\begin{center}
Also iterative, but three-step algorithm\\
Representation, weighting and affectation\\
\end{center}
\end{frame}

\begin{frame}[c]
\frametitle{The B-SOM Algorithm}
\begin{enumerate}
\item Initialization
	\begin{itemize}
	\item Fix: $m$, $\delta$, $K^T$, $N_{iter}$, $T_{min}$, $T_{max}$, $T \leftarrow T_{max}$, $t \leftarrow 0$
	\item Randomly select $m$ distinct prototypes
	\item Set the map $L(m, G^0)$
	\item Assign each object $e_i$ to the closest neuron (cluster)
	\end{itemize}
\end{enumerate}

\end{frame}

\begin{frame}[c]
\frametitle{The B-SOM Algorithm}

\end{frame}



%\section{Dynamic Cluster Algorithm}
%
%\begin{frame}[c]
%\frametitle{Objetivo}
%\begin{block}{}
%Encontrar uma parti��o $ P^* = (C_1, ..., C_k)$ de $E$ em $k$ clusters n�o-vazios e um vetor $L^* = (G_1, ..., G_i, ...G_k)$ tal que $ P^*$ e $L^*$ otimizem o crit�rio:
%\end{block}
%\begin{block}{}
%\begin{center}
%$$\Delta(P^*, L^*) = Min\{\Delta(P,L) / P \in P_k, L \in \Lambda^k\}$$
%\pause
%$$\Delta(P,L) = \sum_{i = 1}^k \sum_{s \in C_i} D(x_s,G_i)$$
%\end{center}
%\end{block}
%\end{frame}
%
%\begin{frame}[c]
%\frametitle{Algoritmo}
%\begin{enumerate}
%\item Inicializa��o: Inicia de uma parti��o aleat�ria $P = (C_1, ..., C_i, ..., C_k)$ ou de um vetor $(G_1, ..., G_i, ..., G_k)$ de $k$ prot�tipos aleat�rios escolhidos entre os elementos de $E$. Neste caso:
%	\begin{itemize}
%	\item $C_i = \emptyset$ para $i = 1, ..., k$
%	\item Para $s = 1$ at� $n$ fa�a:
%		\begin{itemize}
%		\item Atribua $s$ ao cluster $C_l$, $l = argmin_{i = 1, ..., k}D(x_s, G_i)$
%		\item $C_l = C_l \cup \{s\}$
%		\end{itemize}
%	\end{itemize}
%\end{enumerate}
%\end{frame}
%	
%\begin{frame}
%\frametitle{Algoritmo}
%\begin{enumerate}
%\setcounter{enumi}{1}
%\item Etapa de representa��o: $i = 1$ at� $k$, obter o prot�tipo $G_i$ que minimiza o crit�rio:
%$$f_{C_i}(G) = \sum_{s \in C_i} D(x_s,G), G\in\Lambda$$
%
%\item Etapa de aloca��o:
%	\begin{itemize}
%	\item $test \leftarrow 0$
%	\item para $s = 1$ at� $n$ fa�a:
%		\begin{itemize}
%		\item Encontre o cluster $C_m$ ao qual $s$ pertence
%		\item Encontre o �ndice $l$ tal que:  $l = argmin_{i = 1, ..., k}D(x_s, G_i)$
%		\item Se $l \neq m$:\\
%
%			$test \leftarrow 1$\\
%			$C_l = C_l \cup \{s\}$ e $C_m = C_m - \{s\}$\\
%
%		\end{itemize}
%	\end{itemize}
%	
%\item Se $test = 0$ pare, sen�o v� para 2
%\end{enumerate}
%\end{frame}

\end{document} 