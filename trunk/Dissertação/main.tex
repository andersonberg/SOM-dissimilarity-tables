%% RiSE Latex Template - version 0.5
%%
%% RiSE's latex template for thesis and dissertations
%% http://risetemplate.sourceforge.net
%%
%% (c) 2010 Yguarat� Cerqueira Cavalcanti (yguarata@gmail.com)
%%          Vinicius Cardoso Garcia (vinicius.garcia@gmail.com)
%%
%% This document was initially based on UFPEThesis template, from Paulo Gustavo
%% S. Fonseca.
%%
%% ACKNOWLEDGEMENTS
%%
%% We would like to thanks the RiSE's researchers community, the 
%% students from Federal University of Pernambuco, and other users that have
%% been contributing to this projects with comments and patches.
%%
%% GENERAL INSTRUCTIONS
%%
%% We strongly recommend you to compile your documents using pdflatex command.
%% It is also recommend use the texlipse plugin for Eclipse to edit your documents.
%%
%% Options:
%%         * Idiom
%%           pt   - Portuguese (default)
%%           en   - English
%%
%%         * Text type
%%           bsc  - B.Sc. Thesis
%%           msc  - M.Sc. Thesis (default)
%%           qual - PHD qualification (not tested yet)
%%           prop - PHD proposal (not tested yet)
%%           phd  - PHD thesis
%%
%%         * Media
%%           scr  - to eletronic version (PDF) / see the users guide
%%
%%         * Pagination
%%           oneside - unique face press
%%           twoside - two faces press
%%
%%		   * Line spacing
%%           singlespacing  - the same as using \linespread{1}
%%           onehalfspacing - the same as using \linespread{1.3}
%%           doublespacing  - the same as using \linespread{1.6}
%%
%% Reference commands. Use the following commands to make references in your
%% text:
%%          \figref  -- for Figure reference
%%          \tabref  -- for Table reference
%%          \eqnref  -- for equation reference
%%          \chapref -- for chapter reference
%%          \secref  -- for section reference
%%          \appref  -- for appendix reference
%%          \axiref  -- for axiom reference
%%          \conjref -- for conjecture reference
%%          \defref  -- for definition reference
%%          \lemref  -- for lemma reference
%%          \theoref -- for theorem reference
%%          \corref  -- for corollary reference
%%          \propref -- for proprosition reference
%%          \pgref   -- for page reference
%%
%%          Example: See \chapref{chap:introduction}. It will produce 
%%                   'See Chapter 1', in case of English language.

%% M.Sc. Thesis 
%% Author: Anderson Berg
%% CIn-UFPE

\documentclass[pt,msc,oneside,onehalfspacing]{risethesis}

\usepackage{natbib}
\usepackage{babel}
\usepackage{supertabular}
\usepackage{fancybox}
\usepackage{acronym}
\usepackage{multirow}
\newtheorem{proposition}{Proposi��o}[section]
%% Change the following pdf author attribute name to your name.
\usepackage[linkcolor=blue,citecolor=blue,urlcolor=blue,colorlinks,pdfpagelabels,pdftitle={ycc-msc},pdfauthor={Anderson Berg dos S. Dantas}]{hyperref}

\address{RECIFE}

\universitypt{Universidade Federal de Pernambuco}
\universityen{Federal University of Pernambuco}

\departmentpt{Centro de Inform�tica}
\departmenten{Center for Informatics}

\programpt{P�s-gradua��o em Ci�ncia da Computa��o}
\programen{Graduate in Computer Science}

\majorfieldpt{Ci�ncia da Computa��o}
\majorfielden{Computer Science}

\title{SOM para dados relacionais baseados em m�ltiplas tabelas de dissimilaridade}
\date{2012}

\author{Anderson Berg dos Santos Dantas}
\adviser{Francisco de Assis Ten�rio de Carvalho}
%\coadviser{}

\begin{document}

\frontmatter
\frontpage
\presentationpage

\begin{dedicatory}

\end{dedicatory}

\acknowledgements
%I would like to thank and dedicate this dissertation to the following people:
\ldots

%\begin{epigraph}[Open My Eyes]{S.o.ja.}
%I open my eyes each morning I rise, to find a true\\
%thought, know that it's real, I'm lucky to breathe,\\
%I'm lucky to feel, I'm glad to wake up, I'm glad to be\\
%here, with all of this world, and all of it's pain, all\\
%of it's lies, and all of it's flipped down, I still\\
%feel a sense of freedom, so glad I'm around,\\
%\vspace{0.5cm}
%It's my freedom, can't take it from me, i know it, it\\
%won't change, but we need some understanding, I know\\
%we'll be all right.
%\end{epigraph}

\resumo
%Manuten��o e evolu��o de \emph{software} s�o atividades caracterizadas pelo
seu enorme custo e baixa velocidade de execu��o. N�o obstante, elas s�o
atividades inevit�veis para garantir a qualidade do \emph{software} -- quase todo
\emph{software} bem sucedido estimula os usu�rios a fazer pedidos de mudan�as
e melhorias. Sommerville � ainda mais enf�tico e diz que mudan�as em projetos de
\emph{software} s�o um fato. Al�m disso, diferentes estudos t�m afirmado ao
longo dos anos que as atividades de manuten��o e evolu��o de \emph{software} s�o
as mais caras do ciclo de desenvolvimento, sendo respons�vel por cerca de
at� 90\% dos custos.

Todas essas peculiaridades da fase de manuten��o e evolu��o de \emph{software}
leva o mundo acad�mico e industrial a investigar constantemente novas
solu��es para reduzir os custos dessas atividades. Neste contexto, Ger�ncia de
Configura��o de Software (GCS) � um conjunto de atividades e normas para a
gest�o da evolu��o e manuten��o de \emph{software}; GCS define como s�o
registradas e processadas todas as modifica��es, o impacto das mesmas em todo
o sistema, dentre outros procedimentos. Para todas estas tarefas de GCM
existem diferentes ferramentas de aux�lio, tais como sistemas de controle de
vers�o e \emph{bug trackers}. No entanto, alguns problemas podem surgir devido ao uso
das mesmas, como por exemplo o problema de atribui��o autom�tica de respons�vel
por um \emph{bug report} e o problema de duplica��o de \emph{bug reports}.

Neste sentido, esta disserta��o investiga o problema de duplica��o de
\emph{bug reports} resultante da utiliza��o de \emph{bug trackers} em projetos
de desenvolvimento de \emph{software}. Tal problema � caracterizado pela
submiss�o de dois ou mais \emph{bug reports} que descrevem o mesmo problema
referente a um \emph{software}, tendo como principais conseq��ncias a
sobrecarga de trabalho na busca e an�lise de \emph{bug reports}, e o mal
aproveitamento do tempo destinado a essa atividade.

\begin{keywords}
relatos de bug, gerenciadores de relatos de bug, relatos de bug duplicados,
requisi��o de mudan�a, experimento, estudo de caracteriza��o, ferramenta, busca
\end{keywords}

\abstract
%Software maintenance and evolution are characterised by their huge cost and
slow speed of implementation. Yet they are inevitable activities -- almost all
software that is useful and successful stimulates user-generated requests for
change and improvements. \citeauthor{Sommerville2007} is
even more emphatic and says that software changes is a fact of life for large
software systems. In addition, a set of studies has stated along the years that
software maintenance and evolution is the most expensive phase of software
development, taking up to 90\% of the total costs.

All those characteristics from software maintenance lead the academia and
industry to constantly investigate new solutions to reduce costs in such
phase. In this context, Software Configuration Management (SCM) is a set of
activities and standards for managing and evolving software; SCM defines how
to record and process proposed system changes, how to relate these to system
components, among other procedures. For all these tasks it has been proposed
different tools, such as version control systems and bug trackers. However, some
issues may arise due to these tools usage, such as the dynamic assignment of a
developer to a bug report or the bug report duplication problem.

In this sense, this dissertation investigates the problem of bug report
duplication emerged by the use of bug trackers on software development
projects. The problem of bug report duplication is characterized by the
submission of two or more bug reports that describe the same software issue, and
the main consequence of this problem is the overhead of rework when managing
these bug reports.

\begin{keywords}
bug reports, bug trackers, bug report duplication, change request,
tool experiment, bug report duplication characterization study, bug report
search and analysis tool
\end{keywords}

\tableofcontents
\listoffigures
\listoftables
% Acronyms
%\chapter*{List of Acronyms}
\addcontentsline{toc}{chapter}{List of Acronyms}
\begin{acronym}[C.E.S.A.R.]
  \acro{AJAX}{Asynchronous JavaScript and XML}
  \acro{BAST}{Bug Report Analysis and Search Tool}
  \acro{BTT}{Bug Report Tracker Tool}
  \acro{BRN}{Bug Report Network}
  \acro{CCB}{Change Control Board}
  \acro{C.E.S.A.R.}{Recife Center For Advanced Studies and Systems
  \acroextra{C.E.S.A.R. (\url{http://www.cesar.org.br}) is a CMMi level 3
  company with around 700 employees}}
  \acro{FR}{Functional Requirement}
  \acro{GQM}{Goal Question Metric}
  \acro{LOC}{Lines of Code}
  \acro{NFR}{Non-Functional Requirement}
  \acro{NLP}{Natural Language Processing}
  \acro{ORM}{Object-Relational Mapper}
  \acro{RiSE}{Reuse in Software Engineering Group \acroextra{\url{http://www.rise.com.br}}}
  \acro{SCM}{Software Configuration Management}
  \acro{SD}{Standard Deviation}
  \acro{TF-IDF}{Term Frequency-Inverse Document Frequency}
  \acro{UFPE}{Federal University of Pernambuco}
  \acro{VSM}{Vector Space Model}
  \acro{WAD}{Work as Design}
  \acro{XP}{eXtreme Programming}
\end{acronym}
\mainmatter

%% Here you include your chapters
% %% M.Sc. Thesis
%% Author: Anderson Berg
%% CIn-UFPE

\chapter{Mapa auto-organiz�vel}

O mapa auto-organiz�vel faz parte do grupo de modelos de aprendizado n�o-supervisionado de aprendizado competitivo. O objetivo destes modelos � encontrar uma estrutura l�gica entre os dados fornecidos, n�o existe uma resposta esperada nem uma a��o determinada que deva ser realizada. Mapas auto-organiz�veis foram introduzidos por T. Kohonen em 1981. Os primeiros modelos foram projetados para tratar dados de grandes dimens�es. Para realizar esse processamento, a metodologia de visualiza��o topol�gica � projetada para particionar os dados em agrupamentos (\textit{clusters}) que exibem alguma similaridade. 

A caracter�stica mais importante dos mapas auto-organiz�veis � a possibilidade de comparar agrupamentos. Cada observa��o � afetada a um agrupamento e cada agrupamento � projetado em um n� do mapa. Observa��es semelhantes s�o projetadas no mesmo n�. A dissimilaridade entre as observa��es projetadas aumenta com a dist�ncia que separa os n�s.

Classificadores n�o-supervisionados e mapas auto-organiz�veis s�o classes
de m�todos que buscam agrupar dados semelhantes. A maioria das aplica��es
que usam mapas auto-organiz�veis s�o classificadores.

Kohonen projetou um algoritmo auto-organiz�vel que projeta dados em grandes
dimensionalidades em um espa�o discreto de baixa dimensionalidade. Este espa�o � composto
de um grafo n�o-orientado que tem, geralmente 1, 2 ou 3 dimens�es, este
grafo � denominado mapa. O mapa � formado por neur�nios interconectados,
as conex�es entre os neur�nios s�o as arestas do grafo. A estrutura de
grafo permite a defini��o de uma dist�ncia inteira $\delta$ no conjunto
$C$ de neur�nios. Para cada par de neur�nios $(c,r)$ do mapa, $\delta(c,r)$
� o tamanho do caminho mais curto em $C$ entre $c$ e $r$. Para qualquer
neur�nio $c$, a dist�ncia permite a defini��o da vizinhan�a de $c$ de
ordem $d$,

\begin{equation}
V_c(d) = {r \in C, \delta(c,r) \leq d}.
\end{equation}


% \chapter{M�todos de agrupamento}

Os m�todos de agrupamento (\textit{clustering}) buscam organizar um conjunto de dados em grupos, tais que objetos dentro de um certo agrupamento possuem alto grau de similaridade, enquanto que objetos de diferente agrupamentos possuem um alto grau de dissimilaridade. Estes m�todos t�m sido largamente utilizados em �reas como taxonomia, processamento de imagem, recupera��o de informa��o e minera��o de dados. As t�cnicas de agrupamento mais populares s�o os m�todos hier�rquico e de particionamento.
%% M.Sc. Thesis
%% Author: Anderson Berg
%% CIn-UFPE

\chapter{Mapa auto-organiz�vel}

O mapa auto-organiz�vel faz parte do grupo de modelos de aprendizado n�o-supervisionado de aprendizado competitivo. O objetivo destes modelos � encontrar uma estrutura l�gica entre os dados fornecidos, n�o existe uma resposta esperada nem uma a��o determinada que deva ser realizada. Mapas auto-organiz�veis foram introduzidos por T. Kohonen em 1981. Os primeiros modelos foram projetados para tratar dados de grandes dimens�es. Para realizar esse processamento, a metodologia de visualiza��o topol�gica � projetada para particionar os dados em agrupamentos (\textit{clusters}) que exibem alguma similaridade. 

A caracter�stica mais importante dos mapas auto-organiz�veis � a possibilidade de comparar agrupamentos. Cada observa��o � afetada a um agrupamento e cada agrupamento � projetado em um n� do mapa. Observa��es semelhantes s�o projetadas no mesmo n�. A dissimilaridade entre as observa��es projetadas aumenta com a dist�ncia que separa os n�s.

Classificadores n�o-supervisionados e mapas auto-organiz�veis s�o classes
de m�todos que buscam agrupar dados semelhantes. A maioria das aplica��es
que usam mapas auto-organiz�veis s�o classificadores.

Kohonen projetou um algoritmo auto-organiz�vel que projeta dados em grandes
dimensionalidades em um espa�o discreto de baixa dimensionalidade. Este espa�o � composto
de um grafo n�o-orientado que tem, geralmente 1, 2 ou 3 dimens�es, este
grafo � denominado mapa. O mapa � formado por neur�nios interconectados,
as conex�es entre os neur�nios s�o as arestas do grafo. A estrutura de
grafo permite a defini��o de uma dist�ncia inteira $\delta$ no conjunto
$C$ de neur�nios. Para cada par de neur�nios $(c,r)$ do mapa, $\delta(c,r)$
� o tamanho do caminho mais curto em $C$ entre $c$ e $r$. Para qualquer
neur�nio $c$, a dist�ncia permite a defini��o da vizinhan�a de $c$ de
ordem $d$,

\begin{equation}
V_c(d) = {r \in C, \delta(c,r) \leq d}.
\end{equation}


%% M.Sc. Thesis
%% Author: Anderson Berg
%% CIn-UFPE

\chapter{Mapa auto-organiz�vel}

O mapa auto-organiz�vel faz parte do grupo de modelos de aprendizado n�o-supervisionado de aprendizado competitivo. O objetivo destes modelos � encontrar uma estrutura l�gica entre os dados fornecidos, n�o existe uma resposta esperada nem uma a��o determinada que deva ser realizada. Mapas auto-organiz�veis foram introduzidos por T. Kohonen em 1981. Os primeiros modelos foram projetados para tratar dados de grandes dimens�es. Para realizar esse processamento, a metodologia de visualiza��o topol�gica � projetada para particionar os dados em agrupamentos (\textit{clusters}) que exibem alguma similaridade. 

A caracter�stica mais importante dos mapas auto-organiz�veis � a possibilidade de comparar agrupamentos. Cada observa��o � afetada a um agrupamento e cada agrupamento � projetado em um n� do mapa. Observa��es semelhantes s�o projetadas no mesmo n�. A dissimilaridade entre as observa��es projetadas aumenta com a dist�ncia que separa os n�s.

Classificadores n�o-supervisionados e mapas auto-organiz�veis s�o classes
de m�todos que buscam agrupar dados semelhantes. A maioria das aplica��es
que usam mapas auto-organiz�veis s�o classificadores.

Kohonen projetou um algoritmo auto-organiz�vel que projeta dados em grandes
dimensionalidades em um espa�o discreto de baixa dimensionalidade. Este espa�o � composto
de um grafo n�o-orientado que tem, geralmente 1, 2 ou 3 dimens�es, este
grafo � denominado mapa. O mapa � formado por neur�nios interconectados,
as conex�es entre os neur�nios s�o as arestas do grafo. A estrutura de
grafo permite a defini��o de uma dist�ncia inteira $\delta$ no conjunto
$C$ de neur�nios. Para cada par de neur�nios $(c,r)$ do mapa, $\delta(c,r)$
� o tamanho do caminho mais curto em $C$ entre $c$ e $r$. Para qualquer
neur�nio $c$, a dist�ncia permite a defini��o da vizinhan�a de $c$ de
ordem $d$,

\begin{equation}
V_c(d) = {r \in C, \delta(c,r) \leq d}.
\end{equation}


%% M.Sc. Thesis
%% Author: Anderson Berg
%% CIn-UFPE

\chapter{Resultados}

\section{Base de dados �ris}

\begin{table}[h!]
\begin{center}
\caption{Base de dados �ris: ��ndices $CR$, $F-measure$, e $OERC$} \label{iris_index}
%{\scriptsize
\begin{tabular}{|c|c|c|c|c|} \hline
 �ndices & $T_{max}$ & B-SOM & AB-SOM & global AB-SOM \\ \hline
\multirow{4}{1.8cm}{$CR$} & 6 & 0.4017 & 0.4847 & 0.4599\\ 
 & 7 & 0.3999 & 0.5067 & 0.5157 \\ 
 & 9 & 0.3979 & 0.3978 & 0.4913 \\ 
 & 16 & 0.3958 & 0.4653 & 0.3889 \\ \hline
\multirow{4}{1.8cm}{$F-measure$} & 6 & 0.4931 & 0.5785 & 0.5501 \\
 & 7 & 0.5229 & 0.6071 & 0.5575  \\
 & 9 & 0.5394 & 0.5383 & 0.5490  \\
 & 16 & 0.5340 & 0.5742 & 0.5250 \\ \hline
\multirow{4}{1.8cm}{$OERC$} & 6 & 2.67\% & 4.00\% &  4.67\% \\ 
 & 7 &  2.67\% & 4.67\% & 4.00\% \\ 
 & 9 & 4.67\% & 3.33\% & 5.49\% \\ 
 & 16 & 2.00\% & 4.67\% & 3.33\% \\ \hline

\end{tabular}
%}
\end{center}
\end{table}

\begin{table}[h!]
\caption{Base �ris: Matrizes relevantes ($T_{max} = 16$)} \label{iris_matrices}
\begin{center}
%{\scriptsize
\begin{tabular}{|c|c|c|}
\hline
Modelo & Matriz mais importante & Matriz menos importante \\ \hline
AB-SOM & 3-Petal length & 1-Sepal length \\ \hline
global AB-SOM & 3-Petal length & 2-Sepal width \\ \hline
\end{tabular}
%}
\end{center}
\end{table}

\begin{table}[!h]
\caption{Base �ris: matriz de confus�o do algoritmo AB-SOM para m�tiplas tabelas de dissimilaridade ($T_{max} = 6$)}\label{iris_adapt_matrix}
\begin{center}
%{\scriptsize
\begin{tabular}{|c|c|c|c|}
\hline
\multicolumn{1}{|c|}{} &
\multicolumn{3}{|c|}{Classes}
\\ \cline{2-4}
Agrupamentos & 1-Iris setosa & 2-Iris versicolour & 3-Iris virginica \\ \hline
0,0 & 0 & 16 & 0 \\ \hline
0,1 & 0 & 12 & 0 \\ \hline
0,2 & 0 & 1 & 10 \\ \hline
0,3 & 0 & 4 & 0 \\ \hline
0,4 & 0 & 0 & 12 \\ \hline
0,5 & 0 & 0 & 3 \\ \hline
0,6 & 0 & 0 & 0 \\ \hline
0,7 & 38 & 0 & 0 \\ \hline \hline
1,0 & 0 & 2 & 10 \\ \hline
1,1 & 0 & 9 & 0 \\ \hline
1,2 & 0 & 3 & 0 \\ \hline
1,3 & 0 & 0 & 6 \\ \hline
1,4 & 0 & 0 & 4 \\ \hline
1,5 & 0 & 3 & 3 \\ \hline
1,6 & 0 & 0 & 2 \\ \hline
1,7 & 12 & 0 & 0 \\ \hline
\end{tabular}
%}
\end{center}
\end{table}

\begin{table}[h!]
\caption{Base �ris: Matriz de pesos final do algoritmo AB-SOM para m�ltiplas tabelas de dissimilaridade ($T_{max} = 6$)}\label{iris_absom_weight}
\begin{center}
%{\scriptsize
\begin{tabular}{|c|c|c|c|c|}
\hline
\multicolumn{1}{|c|}{} &
\multicolumn{4}{|c|}{Matriz}
\\ \cline{2-5}
Agrupamentos & 1-Sepal length & 2-Sepal width & 3-Petal length & 4-Petal width\\ \hline
0,0 & 0.5878 & 0.5075 & \textbf{2.1617} & \textbf{1.5507}\\ \hline
0,1 & 0.1438 & 0.7925 & \textbf{1.5381} & \textbf{5.7034}\\ \hline
0,2 & 0.1193 & \textbf{47.9332} & 0.5631 & 0.3104\\ \hline
0,3 & 0.7568 & 0.9139 & \textbf{1.7698} & 0.8169\\ \hline
0,4 & 0.3581 & 0.3508 & \textbf{3.3944} & \textbf{2.3451}\\ \hline
0,5 & \textbf{1.7980} & 0.2266 & \textbf{2.7677} & 0.8866\\ \hline
0,6 & \textbf{1.0037} & 0.3404 & \textbf{1.6919} & \textbf{1.7297}\\ \hline
0,7 & 0.2591 & 0.0672 & \textbf{5.2699} & \textbf{10.8944}\\ \hline
1,0 & 0.4403 & 0.3797 & \textbf{4.1288} & \textbf{1.4484}\\ \hline
1,1 & 0.2676 & 0.2180 & \textbf{4.7063} & \textbf{3.6428}\\ \hline
1,2 & 0.1780 & \textbf{2.1141} & \textbf{1.1214} & \textbf{2.3694}\\ \hline
1,3 & \textbf{1.7573} & 0.0686 & \textbf{3.2489} & \textbf{2.5531}\\ \hline
1,4 & 0.3215 & 0.1939 & \textbf{4.4941} & \textbf{3.5684}\\ \hline
1,5 & \textbf{1.4477} & 0.1896 & \textbf{1.0308} & \textbf{3.5350}\\ \hline
1,6 & \textbf{2.0289} & 0.7747 & 0.6483 & 0.9812\\ \hline
1,7 & 0.4077 & 0.1109 & \textbf{5.3766} & \textbf{4.1132}\\ \hline
\end{tabular}
%}
\end{center}
\end{table}



\section{Base de dados E.coli}

\begin{table}[h!]
\begin{center}
\caption{Base E.coli: �ndices $CR$, $F-measure$, e $OERC$} \label{ecoli_index}

\begin{tabular}{|c|c|c|c|c|} \hline
 �ndices & $T_{max}$ & B-SOM & AB-SOM & global AB-SOM \\ \hline
\multirow{4}{1.8cm}{$CR$} & 4 & 0.3035 & 0.3213 & 0.2739 \\ 
 & 4.5 & 0.3462 & 0.2899 & 0.3165 \\ 
 & 6 & 0.3264 & 0.2867 & 0.3555 \\ 
 & 10 & 0.3303 & 0.2998 & 0.2925 \\ \hline
\multirow{4}{1.8cm}{$F-measure$} & 4 & 0.4473 & 0.4629 & 0.4082 \\
 & 4.5 & 0.5486 & 0.4153 & 0.4680 \\
 & 6 & 0.4631 & 0.4036 & 0.5265 \\
 & 10 & 0.5187 & 0.4385 & 0.4460 \\ \hline
\multirow{4}{1.8cm}{$OERC$} & 4 & 21.43\% & 17.56\% & 24.40\% \\ 
 & 4.5 & 16.37\% & 25.30\% & 20.54\% \\ 
 & 6 & 16.37\% & 24.11\% & 14.88\% \\ 
 & 10 & 15.77\% & 24.70\% & 24.40\% \\ \hline

\end{tabular}
\end{center}
\end{table}

\begin{table}[h!]
\caption{Base E.coli: Matrizes relevantes ($T_{max} = 4$)} \label{ecoli_matrices}
\begin{center}

\begin{tabular}{|c|c|c|}
\hline
Modelo & Matriz mais importante & Matriz menos importante \\ \hline
AB-SOM & 5 & 3\\ \hline
global AB-SOM & 4 & 3\\ \hline
\end{tabular}
\end{center}
\end{table}

\begin{table}[h!]
\caption{Base E.coli: matriz de confus�o do algoritmo AB-SOM global para m�ltiplas tabelas de dissimilaridade ($T_{max} = 10$)}\label{ecoli_adapt_matrix}
\begin{center}

\begin{tabular}{|c|c|c|c|c|c|c|c|c|}
\hline
\multicolumn{1}{|c|}{} &
\multicolumn{8}{|c|}{Classes}
\\ \cline{2-9}
Agrupamentos & 1 & 2 & 3 & 4 & 5 & 6 & 7 & 8 \\ \hline
0,0 & 0 & 18 & 0 & 0 & 16 & 1 & 0 & 0 \\ \hline
0,1 & 0 & 10 & 0 & 1 & 24 & 0 & 0 & 0 \\ \hline
0,2 & 0 & 6 & 1 & 0 & 17 & 1 & 0 & 0 \\ \hline
0,3 & 6 & 0 & 0 & 0 & 1 & 4 & 0 & 0 \\ \hline
0,4 & 22 & 0 & 0 & 0 & 2 & 0 & 0 & 0 \\ \hline \hline
1,0 & 1 & 1 & 0 & 1 & 6 & 5 & 5 & 2 \\ \hline
1,1 & 0 & 0 & 0 & 0 & 10 & 0 & 0 & 0 \\ \hline
1,2 & 10 & 0 & 0 & 0 & 0 & 1 & 0 & 0 \\ \hline
1,3 & 15 & 0 & 0 & 0 & 0 & 1 & 0 & 0 \\ \hline
1,4 & 22 & 0 & 0 & 0 & 0 & 0 & 0 & 0 \\ \hline \hline
2,0 & 2 & 0 & 1 & 0 & 0 & 29 & 0 & 9 \\ \hline
2,1 & 0 & 0 & 0 & 0 & 0 & 10 & 0 & 9 \\ \hline
2,2 & 6 & 0 & 0 & 0 & 1 & 0 & 0 & 0 \\ \hline
2,3 & 25 & 0 & 0 & 0 & 0 & 0 & 0 & 0 \\ \hline
2,4 & 34 & 0 & 0 & 0 & 0 & 0 & 0 & 0 \\ \hline
\end{tabular}
\end{center}
\end{table}

\begin{table}[h!]
\caption{Base E.coli: Matriz de pesos final do algoritmo AB-SOM global para m�ltiplas tabelas de dissimilaridade ($T_{max} = 10$)}\label{ecoli_gl_absom_weight}
\begin{center}

\begin{tabular}{|c|c|c|c|c|}
\hline
\multicolumn{5}{|c|}{Matriz}\\ \cline{1-5}
1 & 2 & 3 & 4 & 5\\ \hline
0.7636 & 0.7101 & 0.4876 & \textbf{2.1144} & \textbf{1.7887}\\ \hline

\end{tabular}

\end{center}
\end{table}



\section{Base de dados Thyroid}

\begin{table}[h!]
\begin{center}
\caption{Base de dados Thyroid: �ndices $CR$, $F-measure$ e $OERC$} \label{thyroid_index}

\begin{tabular}{|c|c|c|c|c|} \hline
 �ndices & $T_{max}$ & B-SOM & AB-SOM & AB-SOM global\\ \hline
\multirow{4}{1.8cm}{$CR$} & 4 & 0.3662 & 0.4539 & 0.5700 \\
 & 5 & 0.3642 & 0.3689 & 0.3604 \\ 
 & 6 & 0.2618 & 0.3183 & 0.5882 \\ 
 & 10 & 0.3784 & 0.3660 & 0.3654 \\ \hline
\multirow{4}{1.8cm}{$F-measure$} & 4 & 0.4961 & 0.5841 & 0.5978 \\
 & 5 & 0.4870 & 0.4788 & 0.4678 \\
 & 6 & 0.4019 & 0.4850 & 0.6401 \\
 & 10 & 0.5023 & 0.5048 & 0.4885 \\ \hline
\multirow{4}{1.8cm}{$OERC$} & 4 & 9.77\% & 3.25\% & 4.65\% \\ 
 & 5 & 9.30\% & 5.58\% & 4.19\% \\ 
 & 6 & 11.16\% & 4.19\% & 7.44\% \\ 
 & 10 & 9.30\% & 3.72\% & 5.12\% \\ \hline

\end{tabular}
\end{center}
\end{table}

\begin{table}[h!]
\caption{Base de dados Thyroid: Matrizes ($T_{max} = 4$)} \label{thyroid_matrices}
\begin{center}

\begin{tabular}{|c|c|c|}
\hline
Modelo & Matriz mais importante & Matriz menos importante \\ \hline
AB-SOM & 4-TSH & 1-T3-resin uptake test\\ \hline
global AB-SOM & 5-maximal absolute difference in TSH & 1-T3-resin uptake test\\ \hline
\end{tabular}

\end{center}
\end{table}


\section{Base de dados Wine}

\begin{table}[h!]
\begin{center}
\caption{Base de dados Wine: �ndices $CR$, $F-measure$ e $OERC$} \label{wine_index}

\begin{tabular}{|c|c|c|c|c|c|c|} \hline
 �ndices & $T_{max}$ & B-SOM & AB-SOM & global AB-SOM \\ \hline
\multirow{4}{1.8cm}{$CR$} & 4 & 0.2775 & 0.3560 & 0.3498 \\
 & 5 & 0.2857 & 0.3449 & 0.3248 \\ 
 & 6 & 0.2967 & 0.3589 & 0.3489 \\ 
 & 10 & 0.2940 & 0.3773 & 0.3339 \\ \hline
\multirow{4}{1.8cm}{$F-measure$} & 4 & 0.4163 & 0.4897 & 0.4737 \\
 & 5 & 0.4286 & 0.4735 & 0.4357 \\
 & 6 & 0.4343 & 0.5245 & 0.4801 \\
 & 10 & 0.4427 & 0.5332 & 0.4637 \\ \hline
\multirow{4}{1.8cm}{$OERC$} & 4 & 26.40\% & 7.30\% & 2.25\% \\ 
 & 5 & 26.97\% & 3.37\% & 4.49\% \\ 
 & 6 & 26.97\% & 4.49\% & 5.06\% \\ 
 & 10 & 26.97\% & 6.17\% & 6.74\% \\ \hline

\end{tabular}

\end{center}
\end{table}


\begin{table}[h!]
\caption{Base de dados Wine: Matrizes relevantes $T_{max} = 4$} \label{wine_matrices}
\begin{center}

\begin{tabular}{|c|c|c|}
\hline
Modelo & Matriz mais importante & Matriz menos importante \\ \hline
AB-SOM & 2 & 3\\ \hline
global AB-SOM & 7 & 3\\ \hline
\end{tabular}

\end{center}
\end{table}

\bibliographystyle{natbib}
\addcontentsline{toc}{chapter}{Bibliography}
%% Here you put your bibliography file
\bibliography{bibliografia}

% Appendix
%\clearpage
%\addappheadtotoc
%\appendix
%\appendixpage
%% Here you include your appendix files
% \include{appendix/experiment-instruments}

\end{document}
