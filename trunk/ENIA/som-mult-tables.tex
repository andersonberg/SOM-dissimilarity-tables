\section{Self-organizing map para m�ltiplas tabelas de dissimilaridade}
Seja $E = \{e_1,\dots,e_n\}$ um conjunto de $n$ objetos e seja $p$ o n�mero de matrizes de dissimilaridade $\textbf{D}_j = [d_j(e_i,e_l)] (j = 1,\dots,p)$, onde $d_j(e_i,e_l)$ � a similaridade entre os objetos $e_i e e_l (i,l = 1,\dots,n)$ na matriz de dissimilaridades $\textbf{D}_j$. Assumindo que o prot�tipo $G_l$ do \textit{cluster} $C_l$ � um subconjunto de cardinalidade fixa $1 \leq q \ll n$ do conjunto de objetos $E$, isto �, $G_l \in E^{(q)} = \{A \subset E : |A| = q\}$.

\begin{equation}
E = \sum_{i=1}^n \sum_{l=1}^c \exp \{- \frac{(\delta((f(e_i)),l))^2}{2T^2} \sum_{j=1}^p \sum_{e \in G_l} d_j(e_i, e)
\end{equation}

\subsection{O algoritmo}

\begin{enumerate}
\item \textit{Inicializa��o\\}
Fixe o n�mero $c$ de clusters;\\
Fixe a cardinalidade $1 \leq q \ll n$ dos prot�tipos $G_l (l = 1,\dots,c)$;\\
Fixe $\delta$;\\
Fixe a fun��o kernel $K$;\\
Fixe o n�mero de itera��es $N_iter$;\\
Fixe $T_{min}$, $T_{max}$; Atribua $T \leftarrow T_{max}$; Atribua $t \leftarrow 0$;\\
Selecione $c$ prot�tipos aleatoriamente $G_l^{(0)} \in E^{(q)}(l = 1,\dots,c$;\\
Configure o mapa $L(c,\textbf{G}^0)$, onde $\textbf{G}^0 = (G_1^{(0)},\dots,G_c^{(0)})$.\\
Cada objeto $e_i$ � afetado ao prot�tipo mais pr�ximo com o objetivo de obter a parti��o $P^{(0)} = (P_1^{(0)},\dots,P_c^{(0)})$ de acordo com o seguinte crit�rio:
\begin{equation}
(f(e_i))^{(0)} = argmin_{1\leq r\leq c}\sum_{l=1}^c \exp \{- \frac{(\delta(r,l))^2}{2T^2}\} \sum_{j=1}^p \sum_{e \in G_l^{(0)}} d_j(e_i, e)
\end{equation}

\item \textit{Passo 1: Representa��o}\\
Atribua $t = t + 1$;\\
Atribua $T = T_{max} (\frac{T_{min}}{T_{max}})^\frac{t}{N_{iter} - 1}$\\
A parti��o $P^{(t-1)} = (P_1^{(t-1)},\dots,P_c^{(t-1)})$ � fixa.\\
Calcule o prot�tipo $G_l^{(t)} = G^* \in E^{(q)}$ do cluster $P_l^{(t-1)} (l = 1,\dots,c)$ de acordo com a equa��o:

\begin{equation}
G^* = argmin_{G \in E^{(q)}} \sum_{i=1}^n \exp \{- \frac{(\delta((f(e_i))^{(t-1)},l))^2}{2T^2} \sum_{j=1}^p \sum_{e \in G} d_j(e_i, e)
\end{equation}

\item \textit{Passo 2: Afeta��o}\\
Os prot�tipos $G_l^{(t)} \in E^{(q)} (l = 1,\dots,c)$ s�o fixos $P^{(t)} \leftarrow P^{(t-1)}$\\
para $i=1$ at� $n$ fa�a:\\
encontre o cluster $C_m^{(t)}$ ao qual o objeto $e_i$ pertence\\
encontre o cluster vencedor $C_r^{(t)}$ tal que:\\

\begin{equation}
r = (f(e_i))^{(t)} = argmin_1\leq h\leq c \sum_{l=1}^c \exp \{- \frac{(\delta(h,l))^2}{2T^2}\} \sum_{j=1}^p \sum_{e \in G_l^{(t)}} d_j(e_i, e)
\end{equation}

se $r \neq m:\\$
$C_r^{(t)} \leftarrow C_r^{(t)} \cup \{e_i\}$\\
$C_m^{(t)} \leftarrow C_m^{(t)} \setminus \{e_i\}$\\

\item \textit{Crit�rio de parada}\\
Se $T == T_{min}$ ent�o PARE; sen�o volte ao passo 1

\end{enumerate}